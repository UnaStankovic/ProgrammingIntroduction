\section{HTML i CSS}
\label{sec:htmlcss}
HTML i CSS predstavljanju neizostavan materijal pri učenju web programiranja. Upravo zbog njihovog velikog značaja im posvećujem celo poglavlje. Slajdovi korišćeni prilikom predavanja su apsolutno nedovoljan materijal. Svi materijali korišćeni pri kreiranju časova, kao i ovog materijala su javno dostupni i nalaze se na:
\begin{itemize}
\item Aleksandar Veljković, Veb programiranje, Matematički fakultet http://poincare.matf.bg.ac.rs/~aleksandar/files/web/skripta.pdf
\item W3Schools: https://www.w3schools.com/
\item Filip Marić, Uvod u Veb i Internet tehnologije, Matematički fakultet 
\end{itemize}

\subsection{HTML}
\subsubsection{Zadaci sa časa}
Neki od zadataka rađenih na času su navedeni u nastavku.

\begin{primer}
Prva HTML stranica: Treba kreirati svoju prvu HTML stranicu korišćenjem tagova. Propozicije stranice su sledeće:
\begin{itemize}
\item Kreirati naslov stranice i iskoristiti barem 3 nivoa heading-a
\item Napisati neki pasus koji ima smisla, u okviru kog treba iskoristiti break, strong i em tagove
\item Potom kreirati isto to za ostale nivoe headinga
\item Kreirati jos barem 3 stranice, od čega: 2 treba da sadrže smisleni tekst u paragrafima, a poslednja može sadržati tekst u vidu \textit{lorem ipsum dolor sit...}
\item Druga strana treba da sadrži linkove ka barem 3 spoljašnje strane, npr: ujutru uz kafu volim da čitam, pa linkove ka nekoliko portala
\item Svaka strana treba da sadrži veze ka svim ostalim stranama
\end{itemize}
\end{primer}

\begin{primer}
Španska kuhinja: Kreirati sajt prema sledećim propozicijama:
\begin{enumerate}
\item Iz kojih jela se sastoje tipični obroci u Španiji (entrada, primer plato, segundo plato,...)?
\item Kreirati osnovnu stranicu sa opisom delova obroka, a potom za svaki deo obroka kreirati posebnu stranicu.
\item Svaka stranica mora da sadrži
	\begin{itemize}
		\item Naslov, npr.: Primer plato
		\item Podnaslov, naziv nekog tipičnog jela karakterističnog za taj obrok, npr.: Nachos (za entradas)
		\item Pasus, nešto o tom jelu, odakle je poteklo, kad je nastalo  i slično
		\item Pasus sa neuređenom listom potrebnih sastojaka
		\item Pasus sa uređenom listom postupka pripreme
	\end{itemize}
\item Nekoliko slika jela
\item Vezu ka prethodnoj i narednoj stranici
\item Poslednja stranica treba da se vraća na prvu i da ima spoljašnju vezu ka opisu nekog grada u Španiji.
\end{enumerate}
\end{primer}
\begin{primer}
Sportski izveštaj: Kreirati stranicu o sportu. Odabrati sport po izboru, a potom ispuniti specifikaciju:
\begin{enumerate}
\item Napraviti definicionu listu koja opisuje odabrani sport
\item Napraviti listu sa pravilima igre, ako je moguće kreirati listu u listi (ugnježdena lista)
\item Kreirati tabelu sa rezultatima nekog takmičenja
\item Ubaciti dve ili više slika (iskoristiti width i height kako bi se veličina slike prilagodila)
\item Ubaciti tabelu sa rezultatima sa nekog većeg takmičenja iz tog sporta
\end{enumerate}
\end{primer}
\begin{primer}
Stiven Hoking: Kreirati sajt koji će sadržati informacije o Stivenu Hokingu (eng. Stephen Hawking).
Specifikacija sajta je sledeća:
\begin{itemize}
\item Sajt treba da sadrži naslov, kome će u tooltip-u stajati: engleski teoretski fizičar, kosmolog, autor i direktor istaživanja u Centru za teorijsku kosmologiju na Univerzitetu u Kembrdžu.
\item Ispod naslova treba da stoji kratki paragraf o Stivenu Hokingu, paragraf treba da sadrži boldovan i italic tekst, kao i nešto što bi bilo highlightovano i precrtano
\item Kreirati listu sa spiskom njegovih publikacija
\item Ubaciti sliku, i skalirati je korišćenjem odgovarajućih atributa
\item Ubaciti neki citat na odgovarajući način
\item Ubaciti citat nečega sa njegove stranice, i potom referisati stranicu.
\item Ubaciti vezu ka drugoj stranici na kojoj će biti slike i nazivi nekih od njegovih publikacija
\end{itemize}
\end{primer}

\begin{primer}
Forma za registraciju za učešće na nekom kursu. Zadatak je kreirati stranicu koja će sadržati formu, koja bi trebalo da se popuni kako bi se prijavilo za učešće na nekom od kurseva. Sami odaberite nazive kurseva, kao i relevantne informacije polaznika.
\begin{itemize}
\item Ime
\item Prezime
\item Adresa
\item Grad
\item Broj telefona
\item Broj mobilnog
\item e-mail adresa
\item kurs za koji se prijavljuje - lista od barem 5 izbora
\item odabir termina: vikendom, radnim danima ili svejedno - radio buttoni
\item checklista: radim na svom računaru/ potreban mi je računar
\item text area u kojoj treba upisati prethodno iskustvo
\item button za slanje informacija
\end{itemize}
Neke od informacija o polaznicima kurseva su obavezne, neke nisu, sami odredite koje jesu. Osim navedenih treba dodati jos barem 2 dodatne informacije po izboru u različitim oblicima.
\end{primer}

\subsection{CSS}

\subsection{Bootstrap}


\subsubsection{Zadaci sa casa}

\begin{primer}
Kreirati stranicu zdrave hrane.
\end{primer}

\subsubsection{Domaći zadaci}
Domaći zadaci vezani za ovo poglavlje se odnose na unapređivanje i dodavanje sadržaja i isprobavanje tagova i elemenata nad zadacima rađenim na časovima. 

\newpage