\subsection{JQuery biblioteka}

JQuery je jedna od najpopularnijih biblioteka nastala $2006.$ godine koja nam omogućava brojne funkcionalnosti koje umnogome olakšavaju rad. Neke od njih su:
\begin{itemize}
\item HTML/DOM manipulacija
\item CSS manipulacija
\item HTML event metodi
\item Efekti i animacije
\end{itemize} 

Sintaksa $JQuery$-ja sastoji se iz nekoliko elemenata $\$(dohvatanje\ elementa).akcija\ nad\ elementom$. Na primer, $\$('p').hide()$, čime se skrivaju svi pasusi. 
Pre nego što krenemo da pišemo $JQuery$ kod neophodno je da ga uključimo:
\begin{verbatim}
<script src="jquery.js"></script>
\end{verbatim}
Da bismo došli do ove datoteke neophodno je da je skinemo sa zvaničnog sajta. Postoji nekoliko verzija biblioteke, pa treba odabrati onu koja odgovara potrebama.
\\\\
Svaki $JQuery$ kod počinje sa istom linijom koja mora postojati kako bi se sprečilo da se kod izvršava pre nego što su svi $HTML$ elementi učitani:
\begin{verbatim}
$(document).ready(function(){

});
\end{verbatim}
ili kraće sa:
\begin{verbatim}
$(function(){

});
\end{verbatim}

\subparagraph{JQuery selektori}
$JQuery$ selektori predstavljaju jedan od najvažnijih elemenata ove biblioteke. Uz pomoć selektora dohvatamo i baratamo $HTML$ elementima. Svi $JQuery$ selektori počinju sa \$(), a potom pod navodnicima navodimo naziv etikete ili $CSS$ selektor. Veoma je važno napomenuti da je moguće nadovezivati naredbe nakon što se element jednom dohvati ovaj postupak se naziva ulančavanje.
\begin{lstlisting}[backgroundcolor = \color{lightgray}, breaklines=true]
HTML:
<h1>Ovo je naslov</h1>

JS:
$('h1').css({
    'color': 'red',
    'font-size': '20px',
});
\end{lstlisting}
\subparagraph{JQuery događaji}
Događaji su veoma slični nazivima onima u standardnom $JavaScript$-u, a kako bismo ih navodili koristimo $on$. Unutar zagrada, potom, navodimo naziv događaja i funkciju koja treba da se izvrši kada je događaj okinut. Opet, unutar te funkcije možemo dohvatati druge elemente i dodavati neke operacije, na primer: promena boje, dodavanje reagovanja na događaj ili postavljanje animacija. Animacije su nešto što $JQuery$ razlikuje od običnog $JavaScripta$, kao i od drugih biblioteka, jer se ovde na veoma jednostavan način dodaju. Naime, na neki element možemo postaviti animaciju ključnom rečju $animate$ kojoj kao argumente šaljemo osobine elementa (kako se element menja), vreme za koje treba animacija da se izvrši, kao i funkciju koju je potrebno izvršiti nakon što je animacija gotova, Svi navedeni koncepti ilustrovani su na primeru ispod:
\begin{lstlisting}[backgroundcolor = \color{lightgray}, breaklines=true]
HTML:
<h1>Ovo je naslov</h1>
<input type="text" id="polje1">

JS:
$('h1').css({
    'color': 'red',
    'font-size': '20px',
    }).on('click', function(){
        $('#polje1').animate({
            opacity: 0.25,
            left: "+=50",
            height: "toggle"
        }, 500, function() {
             console.log('gotovo');
        });
});
\end{lstlisting}
U $JQuery$-u normalno rade i lambda funkcije:
\begin{lstlisting}[backgroundcolor = \color{lightgray}, breaklines=true]
HTML:
<h1>Ovo je naslov</h1>
<input type="text" id="polje1">

<h2 class="klasa1">123</h2>
<h2 class="klasa1">345</h2>
<h2 class="klasa1">567</h2>
<h2 class="klasa1">678</h2>
<h2 class="klasa1">335</h2>

JS:
$('.klasa1').on('click', () =>
{
    console.log('kliknuto');
});
\end{lstlisting}
Da bismo pokupili vrednost elementa $input$ polja neophodno je da navedemo $.val()$ (prisetimo se u $JavaScript$-u bilo je $.value$). 
\begin{lstlisting}[backgroundcolor = \color{lightgray}, breaklines=true]
$('#polje1').on('input', (e) => {
    console.log($('#polje1').val());
});
\end{lstlisting}