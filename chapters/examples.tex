\section{Zadaci sa casa}
Napomena: Rešenja zadataka nisu jedinstvena. Skoro svi zadaci se mogu uraditi na više načina. Za vežbu pokušati pronaći ili uraditi zadatke na više načina. 

\subsection{JavaScript}

\subsubsection{Uvodni primeri}
\begin{primer}
Napisati program koji za dva cela broja ispisuje najpre
njihove vrednosti, a zatim i njihov zbir, razliku, proizvod, ceo deo pri deljenju
prvog broja drugim brojem i ostatak pri deljenju prvog broja drugim brojem. 
\end{primer}


\begin{primer}
Napisati program koji za realne vrednosti dužina
stranica pravougaonika ispisuje njegov obim i površinu. Ispisati tražene vrednosti
zaokružene na dve decimale. 
\end{primer}

\begin{primer}
Napisati program koji za tri cela broja ispisuje
njihovu artimetičku sredinu zaokruženu na dve decimale.
\end{primer}

\begin{primer}
Napisati program koji za dva cela broja a i b dodeljuje
promenljivoj rezultat vrednost 1 ako važi uslov:
a) a i b su različiti brojevi
b) a i b su parni brojevi
c) a i b su pozitivni brojevi, ne veći od 100
U suprotnom, promenljivoj rezultat dodeliti vrednost 0. Ispisati vrednost promenljive
rezultat.
\end{primer}

\begin{primer}
Napisati program koji za dva cela broja ispisuje
njihov maksimum.
\end{primer}

\begin{primer}
Napisati program koji za dva cela broja ispisuje
njihov minimum.
\end{primer}

\begin{primer}
Napisati program koji za tri cela broja ispisuje zbir
pozitivnih.
\end{primer}

\begin{primer}
U prodavnici je organizovana akcija da svaki kupac dobije
najjeftiniji od tri artikla za jedan dinar. Napisati program koji za cene
tri artikla izračunava ukupnu cenu, kao i koliko dinara se uštedi zahvaljujući
popustu.
\end{primer}


\begin{primer}
Napisati program koji za redni broj dana u nedelji ispisuje
ime odgovarajućeg dana. U slučaju pogrešnog unosa ispisati odgovarajuću
poruku. Napomena: uraditi zadatak i korišćenjem case
\end{primer}

\begin{primer}
Napisati program koji za uneti karakter ispituje da li je
samoglasnik.
\end{primer}

\begin{primer}
Napisati program koji 5 puta ispisuje tekst $"$Mi volimo da
programiramo$"$.
\end{primer}

\begin{primer}
Napisati program koji preko prompta učitava ceo broj n i ispisuje n puta
tekst $"$Mi volimo da
programiramo$"$.
\end{primer}

\begin{primer}
Napisati program koji učitava pozitivan ceo broj n a potom
ispisuje sve cele brojeve od 0 do n.
\end{primer}

\begin{primer}
Napisati program koji učitava dva cela broja n i m ispisuje
sve cele brojeve iz intervala [n, m].\\
(a) Koristiti while petlju.\\
(b) Koristiti for petlju.\\
(c) Koristiti do-while petlju.\\
\end{primer}

\begin{primer}
Napisati program koji učitava ceo pozitivan broj i izračunava
njegov faktorijal. 
\end{primer}

\begin{primer}
Preko prompta unose se realan broj x i ceo pozitivan
broj n. Napisati program koji izračunava n-ti stepen broja x.
\end{primer}

\begin{primer}
Pravi delioci celog broja su svi delioci sem jedinice i samog
tog broja. Napisati program za ceo pozitivan broj x ispisuje sve njegove
prave delioce.
\end{primer}

\begin{primer}
Promptom se unosi ceo pozitivan broj n, a potom i n celih
brojeva. Izračunati i ispisati zbir onih brojeva koji su neparni i negativni.
\end{primer}

\begin{primer}
Program učitava realan broj x i ceo pozitivan broj n.
Napisati program koji izračunava i ispisuje sumu $S=x + 2*x^2 + 3*x^3 + ..$.
\end{primer}

\begin{primer}
Za unetu pozitivnu celobrojnu vrednost n napisati programe
koji ispisuju odgovarajuće brojeve. Pretpostaviti da je unos korektan.\\
(a) Napisati program koji za unetu pozitivnu celobrojnu vrednost n ispisuje
tablicu množenja.\\
(b) Napisati program koji za uneto n ispisuje sve brojeve od 1 do $n^2$ pri čemu se ispisuje po n brojeva u jednoj vrsti.\\
\end{primer}

\begin{primer}
Program učitava ceo pozitivan broj n, a potom n realnih
brojeva. Odrediti koliko puta je prilikom unosa došlo do promene znaka. Ispisati
dobijenu vrednost.
\end{primer}

\begin{primer}
Kreirati sajt prodavnice u kome svaki kupac određuje sam cenu artikla. U prodavnici se nalazi n artikala čije cene su realni brojevi. Na stranici treba da postoji naziv prodavnice, kao i spisak dostupnih artikala sa slikama, a potom da se uz pomoć prompta učita broj proizvoda koje korisnik želi da kupi. Nakon toga, unose se naziv proizvoda koji korisnik zeli da kupi i cena koju zeli. Za unet broj proizvoda treba u konzoli ispisati naziv proizvoda i ponuđenu cenu, odrediti ukupnu sumu i cenu najjeftinijeg artikla.
\end{primer}

\subsubsection{Niske}
\begin{primer}
Za unetu nisku proveriti da li sadrži broj i ispisati u konzoli. Napomena: koristiti $ASCII$ tabelu.
\end{primer}

\begin{primer}
Zameniti sva pojavljivanja slova $a$ u niski unetoj preko prompta slovom $b$. Na primer, $"$Danas je lep dan.$"$ sa $"$Dbnbs je lep dbn.$"$. Napomena: uraditi kreiranjem nove niske.
\end{primer}

\begin{primer}
Proveriti da li je reč uneta preko prompta palindrom. Palindromi su reči, rečenice koji se mogu čitati unapred i unazad i imaju isto značenje. Primeri za reči: Ana, bob, dovod, kapak, kajak, kuk, melem, neven, oko, pop, potop, ratar, teret. 
\end{primer}

\begin{primer}
Program učitava reč. Napisati
program koji proverava da li se od karaktera unete reči može napisati reč $Zima$.
\end{primer}

\begin{primer}
Program učitava ceo broj n, a zatim i n karaktera. Napisati
program koji proverava da li se od unetih karaktera može napisati reč $Zima$.
\end{primer}

\begin{primer}
Napisati program koji učitava karaktere sve do kraja ulaza (do unosa broja 0), a potom ispisuje broj velikih slova, broj malih slova, broj cifara, broj belina i zbir unetih cifara. 
\end{primer}

\begin{primer}
Napisati program koji za unetu nisku proverava da li sadrži reč $ima$.
\end{primer}

\begin{primer}
Kreirati sajt zdrave hrane. Sajt treba da sadrži naziv, logo, navigacioni bar (O nama, Proizvodi, Kontakt). 
\begin{itemize}
    \item Stranica o nama treba da sadrži: sliku, tekst o radnji, adresu i ostale relevantne informacije. 
    \item Kreirati potom stranicu sa proizvodima koja treba da sadrži najpre listu proizvoda (kategorije, pa potom po nekoliko proizvoda), a ispod toga treba da se nalaze neki od popularnih proizvoda sa slikama i cenama na 100 grama. 
    \item Stranica za kontakt treba da sadrži kontakt formu (Ime, email, prostor za poruku, itd.).  
\end{itemize}

Nakon toga koristeci prompt, alert i confirm za sajt sa zdravom hranom napraviti da se za 5 korisnika unese: ime, ocena sajta i predlog novog naziva zdrave hrane. Proveriti ispravnost unetog podatka, ako podatak nije pravilno unet ponuditi korisniku da ponovo unese podatak. Ako jeste, unosi se naredni. Unos ocena dodati na zbir. U konzoli ispisati predlog korisnika u formatu Ime : predlog. Na kraju, ispisati prosečnu ocenu sajta.
\end{primer}

\subsubsection{Nizovi}

\begin{primer}
Napisati program u kom za niz brojeva program ispisuje niz kvadriranih vrednosti.
\end{primer}

\begin{primer}
Napisati program u kom se unosi n, a potom i n elemenata niza. Nakon toga, na kraj niza se dodaje suma zadatih brojeva.
\end{primer}

\begin{primer}
Napisati program koji računa skalarni proizvod vektora a i b. Vektor $a=(a_1, a_2,...)$ i $b=(b_1, b_2,...)$, a skalarni proizvod $a*b= a_1 * b_1 + a_2 * b_2 + ... + a_n * b_n$. Najpre se unosi n, a potom po n vrednosti za vektore a i b. 
\end{primer}

\begin{primer}
Napisati program koji za dati niz ispisuje:\\
a) elemente na parnim pozicijama u nizu
b) parne elemente niza.
\end{primer}

\begin{primer}
Napisati program koji za učitani ceo broj, ispisuje broj
pojavljivanja svake od cifara u zapisu tog broja.
\end{primer}


\begin{primer}
Napisati program koji transformiše uneti niz tako što
kvadrira sve negativne elemente niza. 
\end{primer}

\begin{primer}
Sa standardnog ulaza se učitava dimenzija niza, elementi
niza i jedan ceo broj k. Napisati program koji štampa indekse elemenata koji su
deljivi sa k.
\end{primer}

\subsubsection{Objekti}
\begin{primer}
Kreirati objekat $"$Auto$"$. Objekat treba da sadrži: marku, godište, cenu.
Ispisati objekat na izlaz.
\end{primer}

\begin{primer}
Kreirati objekat $"$Film$"$, sa imenom filma, godinom izlaska, i ocenom na $IMDB$-u. Kreirati više takvih objekata i smestiti ih u niz (barem $5$). Na izlaz ispisati one filmove koji imaju ocenu preko $8$.  
\end{primer}

\begin{primer}
Kreirati objekat $"$Zaposleni$"$ za neku firmu. Objekat treba da sadrži: ime, prezime, datum rodjenja, identifikacioni broj, poziciju, platu, datum zaposlenja, da li je zaposlen za stalno, pol i spisak iznosa poslednjih $6$ plata.
\end{primer}

\begin{primer}
Za spisak zaposlenih u nekoj firmi (barem $5$), uraditi sledeće:
\begin{itemize}
	\item kreirati spisak zaposlenih: ime i prezime, visina plate, da li su stalno zaposleni
	\item izdvojiti one čija su primanja veća od 30000,
	\item izdvojiti stalno zaposlene,
	\item onima koji nisu stalno zaposleni dodati bonuse u iznosu od 10000,	
	\item proveriti da li se za zaposlene date firme posebno vodi stavka visina poreza i
	\item za zaposlene koji su stalno zaposleni izračunati prosek poslednje 3 plate.
\end{itemize}

\end{primer}

\begin{primer}
Kreirati sajt muzeja. Sajt treba da sadrži navigacioni bar: O nama, Postavke, Karte i Kontakt. Stranica O nama treba da sadrži kratak istorijat muzeja, kao i spisak događaja. Postavke treba da sadrži 3 postavke: Afrička umetnost, Kineska umetnost i Evropska umetnost. Svaka od postavki treba da sadrži bar 3 slike, kratak opis kao i datum početka i završetka postavke. Osim toga treba kreirati inventar za svaku od postavki sa barem 3 stavke. Stranica sa kartama treba da sadrži cene karata prema kategorijama, kao i popuste za specijalne kategorije (mlade do 26 godina, penzionere). Stranica za kontakt treba da sadrži formu: Ime i prezime, E-mail, Poruka.\\\\
U programu kreirati:
\begin{itemize}
\item POSTAVKE: Evropska, Afrička i Kineska. Potom, za svaku od postavki kreirati naziv, datume trajanja (od i do) i sifrarnik inventara ( kao niz sifara).
\item Karte: povlašćene kategorije, standardna cena
\item Kontakt: ime i prezime, e-mail, poruka.
\end{itemize}  
\end{primer}

\subsubsection{Funkcije}

\begin{primer}
Napisati funkciju kvadrat(x) koja računa kvadrat
datog broja. Napisati program koji učitava ceo broj i ispuje rezultat poziva
funkcije.
\end{primer}

\begin{primer}
Napisati funkciju apsolutna\_vrednost(x) koja izračunava apsolutnu vrednost broja x. Napisati program koji učitava jedan ceo broj i ispisuje rezultat poziva funkcije.
\end{primer}

\begin{primer}
Napisati funkciju min(x, y, z) koja izračunava minimum tri broja. Napisati program koji učitava tri cela broja i ispisuje rezultat poziva funkcije.
\end{primer}

\begin{primer}
Napisati funkciju stepen(x, n) koja
računa vrednost n-tog stepena realnog broja x. Napisati program koji učitava
relan broj x i ceo broj n i ispisuje rezultat rada funkcije.
\end{primer}

\begin{primer}
Napisati funkciju faktorijel(n) koja računa
faktorijel broja n. Napisati i program koji učitava dva cela broja x i y iz intervala
[0, 12] i ispisuje vrednost zbira x! + y! (x! znači faktorijel broja x).
\end{primer}

\begin{primer}
Napisati funkciju sadrzi(x, c) koja ispituje
da li se cifra c nalazi u zapisu celog broja x. Funkcija treba da vrati 1 ako se cifra
nalazi u broju, a 0 inače. Napisati program koji učitava dva cela broja i ispisuje
rezultat poziva funkcije.
\end{primer}

\begin{primer}
Napisati program za ispitivanje svojstava cifara datog celog
broja.\\
(a) Napisati funkciju sve\_parne\_cifre koja ispituje da li se dati ceo broj sastoji
isključivo iz parnih cifara. Funkcija treba da vrati 1 ako su sve cifre
broja parne i 0 u suprotnom.\\
(b) Napisati funkciju sve\_cifre\_jednake koja ispituje da li su sve cifre datog
celog broja jednake. Funkcija treba da vrati 1 ako su sve cifre broja jednake
i 0 u suprotnom.\\
Napisati program koji učitava ceo broj i ispisuje da li su sve cifre parne i da li su
sve cifre jednake.
\end{primer}

\begin{primer}
Napisati funkciju rastuce(n) koja ispituje da li
su cifre datog celog broja u rastućem poretku. Funkcija treba da vrati vrednost
1 ako cifre ispunjavaju uslov, odnosno 0 ako ne ispunjavaju uslov. Napisati i
program koji učitava ceo broj i ispisuje poruku da li su cifre unetog broja u
rastućem poretku.
\end{primer}

\begin{primer}
Broj je prost ako je deljiv samo sa 1 i sa samim sobom.
Napisati funkciju int prost (int x) koja ispituje da li je dati ceo broj prost.
Funkcija treba da vrati 1 ako je broj prost i 0 u suprotnom. 
\end{primer}

\begin{primer}
Napisati funkciju int prebrojavanje(float x) koja prebrojava
koliko puta se broj x pojavljuje u nizu brojeva koji se unose sve do pojave
broja 0. Napisati program koji učitava vrednost broja x i testira rad napisane
funkcije.
\end{primer}


\begin{primer}
Napisati funkcije za rad sa nizovima celih brojeva.\\
(a) Napisati funkciju ucitaj(a, n) koja učitava elemente niza a dimenzije n.\\
(b) Napisati funkciju stampaj(a, n) koja štampa elemente niza a dimenzije n.\\
(c) Napisati funkciju suma(a, n) koja računa i vraća sumu
elemenata niza a dimenzije n.\\
(d) Napisati funkciju prosek(a, n) koja računa i vraća prosečnu
vrednost (aritmetičku sredinu) elemenata niza a dimenzije n.\\
(e) Napisati funkciju minimum(a, n) koja izračunava i vraća
minimum elemenata niza a dimenzije n.\\
(f) Napisati funkciju pozicija\_maksimuma(a, n) koja izračunava
i vraća poziciju maksimalnog elementa u nizu a dimenzije n. U slučaju
više pojavljivanja maksimalnog elementa, vratiti najmanju poziciju.\\
(g)Napisati funkciju koja sve vrednosti niza uvećava za zadatu vrednost m.
\end{primer}

\begin{primer}
Kreirati sajt zdrave hrane. Stranica treba da sadrži odgovarajuće HTML i CSS elemente (slike, naslove, tekst, itd.). Korisnik treba da odabere da li želi da učestvuje u anketi ili ne. ako odabere da želi, koristeći ugrađenu funkciju prompt zahtevati od korisnika predlog za naziv zdrave hrane i ocenu sajta. Postupak uraditi za barem 6 korisnika. Nakon što su svi uneli podatke (ili odlučili da ih ne unesu), ispisati spisak svih predloga, a potom ispisati prosečnu ocenu sajta. Uraditi odgovarajuće provere unosa.   
\end{primer}

\subsubsection{Periodično izvršavanje funkcija}
\begin{primer}
Kreirati stranicu-prezentaciju pozorista. Sajt treba da sadrži: naslov, paragraf, slike, deo sa kontakt informacijama i ostale odgovarajuće HTML i CSS elemente. Nakon $10$ sekundi od otvaranja sajta korisnik se obavestava da su tokom januara sve cene karata snižene $50\%$. 
\end{primer}

\begin{primer}
Kreirati stranicu za brzo anketiranje stanovnika. Po otvaranju stranice nakon 3 sekunde izlazi prozor sa pitanjem $"$Da li želite da učestvujete u anketi?$"$. Ako je odgovor potvrdan, korisniku se svake 4 sekunde nudi po jedno od 15 pitanja (pitanja idu redom). Pitanja:
\begin{itemize}
\item Koliko imate godina?
\item Koji ste nivo obrazovanja stekli?
\item Da li ste pušač?
\item Da li jedete meso?
\item Da li redovno odlazite na zdravstvene preglede?
\item Da li radite?
\item Koliko puta nedeljno trenirate?
\item Posecujete li muzeje?
\item Posecujete li pozorište?
\item Imate li partnera?
\item Koliko stranih jezika govorite?
\item Imate li kucnog ljubimca?
\item Koristite li računar svakodnevno?
\item Koliko sati dnevno gledate TV?
\item U kom mestu živite?
\end{itemize}  
Anketiranje traje ukupno minut. Po završetku ankete treba da iskoči prozor sa $"$Hvala na izdvojenom vremenu!$"$. Na kraju, u konzoli se ispisuje spisak pitanja i korisnikovih odgovora. 
\end{primer}

\subsection{Interakcija sa DOM-om iz jezika JavaScript}

\subsubsection{Osnove}
\begin{primer}
Kreirati stranicu bioskopa koja sadrži $4$ $div$a. Prvi $div$ sadrzi naslov stranice i logo. Drugi $div$ sadrži sadržaj stranice (content, na primer, najave premijera). Treci $div$ sadrzi informacije poput kontakt telefona i ostalih informacija o cenama karata i datumima nekih od projekcija. Cetvrti div je footer u kome mogu biti smestene adresa preduzeca kao i polje za prijavu za $newsletter$. Pristupiti pasusu unutar drugog $div$-a promeniti mu pozadinu sa crvene na sivu. Dohvatiti sve naslove $h2$ i promeniti im boju na belu. Postaviti boju elementa sa $id$-jem "neki\_odabrani" na plavu. Za liste postaviti veličinu slova na $11px$. 
\end{primer}

\begin{primer}
Dohvatiti sve elemente liste unutar elementa sa $id$-jem $cenovnik$. Podesiti dohvaćenim elementima slova da budu iskošena, i promeniti boju na belo. Promeniti pozadinsku boju $cenovnika$ na ljubičasto.
\end{primer}

\begin{primer}
Za kreiranu stranicu koja sadrzi 10 pasusa, postaviti pozadinsku boju svakom drugom na svetlo sivu.
\end{primer}

\begin{primer}
Za stranicu sa bioskopom kreirati novi element koji u sebi sadrži 6 $div$-ova sa najavama budućih projekcija. Svaka od najava treba da sadrži naslov filma, poster, datum premijere, kratak opis i link ka detaljnijim informacijama. Treba promeniti boju prvom, poslednjem, kao i drugom i pretposlednjem elementu na: plavu (prvi i poslednji), svetlo plavu (drugi i pretposlednji). Preostala dva elementa treba da ostanu bela. Napomena: slobodno odabrati boje iz palete boja koje bi se uklopile sa vasim sajtom.   
\end{primer}

\begin{primer}
Na stranicu sa bioskopom postaviti tooltip sa tekstom $"$Gledajmo budućnost zajedno!$"$ na glavni naslov. Kreirati element koji sadrži naslov $"$Najpopularniji film u $2018$.$"$, poster i naziv filma. Postaviti atribute širine i visine slike koja predstavlja najpopularniji film na $100$ prema $150$px. Dohvatiti atribute vezane za linkove ka narednim projekcijama iz elementa koji sadrži najave i promeniti ih tako da predstavljaju linkove ka $IMDB$ stranicama filmova. Na kraju, za najpopularniji film, dodati ocenu i link ka $IMDB$ stranici.  
\end{primer}

\subsubsection{Kreiranje elemenata}
\begin{primer}
Na stranicu sa pozorištem dodati nove elemente. Elemente dodavati kreiranjem novih elemenata kroz $JavaScript$. Obavezno dodati tabelu, sliku, link, dok su ostali elementi po izboru u zavisnosti od rada.
\end{primer}

\begin{primer}
Kreirati stranicu na kojoj se nalazi samo jedan prazan $div$ sa $id$-jem $omotac$. U taj $div$, dodavati elemente korišćenjem funkcije $createElement$ dok se ne dobije stranica koja sadrži rezultate ispita. Stranica treba da sadrži barem naslov i tabelu sa spiskom učenika i njihovim ocena koji se izvlači iz objekta.
\end{primer}

\begin{primer}
Za uneti ceo broj $n$, izgenerisati tablicu množenja veličine $n$ koristeći table element i dodavanje uz pomoć interakcije sa DOM drvetom. Doterati stranicu koristeći CSS.
\end{primer}

\begin{primer}
Kreirati sajt koji će sadržati biografije do 3 znamenita naučnika. Na osnovu naziva slike koji se unosi preko prompta, dodati na stranicu primer biografije zadate ličnosti. Na primer, za naziv slike $tesla.jpg$ na stranicu dodati naslov Nikola Tesla, paragraf sa kratkom biografijom, sliku, kao i link ka stranici na Wikipediji. Svaka biografija treba da bude prigodno uokvirena. Ponuditi unos za do 3 licnosti, tako sto ce se preko prompta najpre uneti broj licnosti, a potom i nazivi slika koje korisnik želi da iskoristi pri kreiranju stranice. Sajt ukrasiti korišćenjem prigodinih HTML i CSS elemenata.
\end{primer}

\begin{primer}
Kreirati stranicu koja vrši heširanje tajne poruke koju korisnik unosi preko prompta. Heširanje predstavlja postupak sakrivanja poruke. U ovom zadatku heš treba predstaviti preko tabele, koja u vidu niza predstavlja poruku uz pomoć crvenih, plavih i zelenih kvadratića dimenzija 10x10 piksela. Heš funkcija je sledeća:
\begin{itemize}
\item ako je ostatak pri deljenju kodnog broja karaktera sa 3 jednak 1 onda zelena boja,
\item ako je ostatak pri deljenju kodnog broja karaktera sa 3 jednak 2 onda plava boja
\item inače, crvena. 
\end{itemize}
Prikaz programa dat je na slici \ref{fig:hes}.
\begin{figure}[h!]
\begin{center}
\includegraphics[scale=0.5]{./pictures/hes.png}
\end{center}
\caption{Prikaz programa.}
\label{fig:hes}
\end{figure}

\end{primer}

\subsubsection{Reagovanje na događaje}

\begin{primer}
Kreirati sajt "Mala škola programiranja". Sajt treba da sadrži:
\begin{itemize}
\item pozadinsku sliku
\item navigacioni bar (pocetna kursevi kontakt)
\item na početnoj strani treba da se nalazi malo polje sa pitanjem $"$Da li volite da
     programirate?$"$, a ispod toga dva button-odgovora na pitanje $DA$ i $NE$ 
\item klikom na dugme DA izbacuje se alert sa natpisom "Svaka cast! Buducnost je Vasa!"
\item prelaskom miša preko dugmeta NE, dugme treba da pobegne na proizvoljni deo ekrana
    (koristiti funkciju Random ---> Math.Random())
\item sajt je neophodno ukrasiti odgovarajucim CSS elementima. 
\end{itemize}
\end{primer}

\begin{primer}
Kreirati igru $X$-$O$. Igra se sastoji iz table $3$x$3$. Naizmenično se unose $X$ i $O$ (klikom) dok se u bilo kom smeru ne nađu tri uzastopna ista znaka. Na kraju se ispisuje obaveštenje o tome koji igrač je pobedio i $"$Nerešeno!$"$ ako je rezultat nerešen.
\end{primer}


\begin{primer}
Skočko - Kreirati igru koja je poput igre skočko iz kviza $"$Slagalica$"$. Nasumično se generišu 4 simbola od 6 mogućih. Potom se korisniku nudi da unese odgovarajuće
\end{primer}

\begin{primer}
Kreirati stranicu koja se sastoji iz 4 raznobojna dugmeta kreirana kroz CSS. Stranica predstavlja kulturni vodič kroz grad. Stranica treba da sadrži naslov $"$Kulturni vodič kroz grad$"$, a potom ispod treba da piše odaberite tip događaja. Prelaskom kursora preko dugmeta treba da se promeni boja dugmeta u svetliju nijansu dodeljene boje. Dugmici treba da sadrže naslove: bioskop, pozoriste, muzej, koncert. Klikom na dugme otvara se odgovarajuća stranica. Za stranicu o muzeju modifikovati stranicu sa biografijama najvećih srpskih naucnika tako da ona predstavlja deo prezentacije muzeja. Stranicu o koncertima treba kreirati korišćenjem niza događaja koji bi dodavali različite elemente stranice. 
\end{primer}

\subsubsection{Obrada formulara}

\begin{primer}
Kreirati $"$Mirror$"$ sajt za izvrtanje korisnickog unosa. Pozadinska slika bi trebalo da bude
ogledalo. Kucanjem teksta u input polje istovremeno u paragrafu ispod se ispisuje tekst u 
obrnutom poretku. Sajt ukrasiti odgovarajucim CSS elementima.
\end{primer}

\begin{primer}
Kreirati stranicu na kojoj se unete rečenice (kroz textarea) sortiraju po dužini i ispisuju.
\end{primer}

\begin{primer}
Kreirati igru vešala. Najpre se unosi skrivena reč, na osnovu koje se formiraju polja. Potom se na neki od dva načina nudi korisniku da odabere slovo:
\begin{itemize}
\item unosom kroz polje ili
\item klikom na polje sa slovom.
\end{itemize}

\end{primer}

\begin{primer}
Modifikovati prethodni zadatak, tako da se korisniku ponudi najpre odabir lakše ili teže varijante igre. Ako je odabrana lakša verzija igra se odigrava kao u prethodnom zadatku.  Ako je odabrana teža verzija igra se modifikuje tako da se korisnik stavlja u vremensko ograničenje od 30, 45 ili 60 sekundi i broj pokušaja mu se ograničava na polovinu dužine reči. To znači da do kraja igre dolazi:
\begin{itemize}
\item kada je korisnik pogodio skrivenu reč,
\item kada je isteklo vreme ili
\item kada je istekao broj pokušaja.
\end{itemize}
\end{primer}

\begin{primer}
Kreirati sajt za iznajmljivanje prostora za proslavu rodjendana. Sajt treba da sadrži odgovarajući
 navigacioni bar: o nama, ponuda, galerija I kontakt. 
 Kreirati funkciju za obračunavanje cene usluge po satu. 
 Funkcija treba da sadrži nekoliko argumenata:
\begin{itemize}
 \item broj sati (podrazumevano 2), 
 \item broj dece (podrazumevano 10),
 \item broj odraslih (podrazumevano 5),
 \item ketering (podrazumevano "da"). 
\end{itemize}  
Parametre menjati kroz pozive funkcije za različite argumente. Dopuniti zadatak tako da se korisniku
ponudi da preko forme, klikom na $submit$ dugme (reagovanjem na $click$) omogući formiranje cene 
na osnovu njegovih potreba. Konacnu cenu ispisati na strani.  
Cena se formira po formuli:
     $(brojsati*0.5*brojdece + brojsati*0.7*brojodraslih) *150$, 
u slučaju da je ketering isključen, smanjiti cenu za ukupan broj osoba * $20$. 
\end{primer}

\begin{primer}
Zadatak sa biografijama (naucnicima) oplemeniti i umesto prompta uz pomoc eventova
pokupiti odgovarajuce informacije (broj naučnika, nazive slika), a potom kreirati stranicu. Dozvoljeno je kreiranje stranice na kojoj ce se nalaziti buttoni/input polja i slicno i tako pokupiti broj 1, 2 ili 3 i naziv slike/naucnika.
\end{primer}

\begin{primer}
Kreirati zdravstvenu stanicu. Klikom na dugme dodaj pacijenta izbacuje se forma za dodavanje novog pacijenta. Podatke o pacijentima smeštati kao niz objekata, sa sledećim svojstvima:
\item Ime
\item Prezime
\item Datum rođenja
\item JMBG (13 cifara)
\item Krvna grupa (A, B, AB, 0)
\item RH faktor (pozitivna, negativna).
Klikom na dugme sačuvaj informacije, unete informacije se smeštaju u odgovarajući 
objekat. Omogućiti izlistavanje svih pacijenata, kroz tabelu: ime, prezime i krvna grupa.
Proveriti ispravnost svih unosa.
\end{primer}

\begin{primer}
Dopuniti zadatak sa zdravstvenom stanicom na sledeći način. 
Kreirati pored dugmeta koje izbacuje spisak svih pacijenata dugme koje vrši filtraciju podataka o pacijentima.
Klikom na dugme filteri izbacuje se nekoliko mogućih opcija kroz select element za koje, u zavisnosti da li je opcija obeležena vrši se odgovarajuće filtriranje.
Dodatne parametre filtera uneti pored kroz input polje.
Omogućiti naredne opcije:
\begin{itemize}
\item izdvajanje određene krvne grupe
\item odabir RH faktora
\item izdvajanje svih pacijenata starijih od 30 godina
\item izdvajanje svih pacijenata sa nekim imenom 
\end{itemize}
Klikom na dugme "Filtriraj" izbacuje se spisak pacijenata sa brojevima kartona koji ispunjavaju predviđene uslove. 
U slučaju da je potrebno prebrojati pacijente, to se omogućava uz pomoć check boxa prebroj. Ukupan broj pacijenata se ispisuje ispod spiska.
Npr.: Korišćenjem novih koncepata (filter, map i reduce), izračunati broj pacijenata sa A negativnom krvnom grupom.
\end{primer}

\subsection{Dodatni zadaci}
\begin{primer}
Kreirati stranicu zdrave hrane. Koristeći ugrađenu funkciju prompt zahtevati od korisnika količinu badema koju želi da kupi(u gramima), a potom u alert prozoru ispisati konačnu cenu za unetu količinu. Npr. ako je badem 2000 dinara po kilogramu, a korisnik želi 500 grama, u alertu treba da se ispiše 1000 dinara.  
\end{primer}

\begin{primer}
Kreirati sajt za organizovanje proslava. Sajt treba da sadrži navigacioni bar sa stavkama: o nama, rođendani, proslave jubileja, svadbe i krštenja, kao i kontakt stranu. Rođendanske proslave se sastoje iz više kategorija, kao što su proslave punoletstva, proslave prvih rodjendana i proslave rođendana. Svaka od stavki navigacije vodi ka novim stranicama koje sadrže paragraf sa kratkim opisom usluga koje se nude, tabele sa cenama usluga (Ketering, Premium Ketering, Piće (osnovna i premium ponuda), dekoracija, muzika, transport do lokacije itd.). Svaka od strana sadrži i barem 3 moguće lokacije za proslave, sa sumom cena.

\end{primer}


\subsection{Domaći zadaci}
\begin{primer}
Na koje $if$ se odnosi $else$?
\begin{verbatim}
if (izraz1)
    if (izraz2)
        naredba1
else
    naredba2
\end{verbatim}
\end{primer}
\begin{primer}
Na koje $if$ se odnosi $else$?
\begin{verbatim}
if (izraz1) {
   if (izraz2)
      naredba1
} else
    naredba2
\end{verbatim}
\end{primer}

\begin{primer}
Napisati program koji za uneti pozitivan trocifreni broj na
standardni izlaz ispisuje njegove cifre jedinica, desetica i stotina
\end{primer}

\begin{primer}
Napisati program koji za uneti pozitivan četvorocifreni
broj:\\
(a) izračunava proizvod cifara\\
(b) izračunava razliku sume krajnjih i srednjih cifara\\
(c) izračunava sumu kvadrata cifara\\
(d) izračunava broj koji se dobija ispisom cifara u obrnutom poretku\\
(e) izračunava broj koji se dobija zamenom cifre jedinice i cifre stotine\\
\end{primer}

\begin{primer}
Broj je Armstrongov ako je jednak zbiru kubova svojih
cifara. Napisati program koji za dati trocifren broj proverava da li je Armstrongov.
\end{primer}

\begin{primer}
Napisati program koji ispisuje proizvod parnih cifara unetog
četvorocifrenog broja.
\end{primer}

\begin{primer}
Napisati program koji za ceo broj x ispisuje njegov
znak, tj da li je broj jednak nuli, manji od nule ili veći od nule.
\end{primer}

\begin{primer}
Napisati program koji za uneti prirodan broj ispisuje da li
je on deljiv sumom svojih cifara.
\end{primer}

\begin{primer}
Promptom se unosi ceo pozitivan broj n, a potom i n celih
brojeva. Izračunati i ispisati zbir onih brojeva koji su parni.
\end{primer}

\begin{primer}
Program učitava cele brojeve sve do unosa broja nula 0.
Napisati program koji izračunava i ispisuje aritmetičku sredinu unetih brojeva na
četiri decimale.
\end{primer}

\begin{primer}
U prodavnici se nalaze artikli, čije cene su realni pozitivni
brojevi. Program unosi cene artikala sve do unosa broja nula 0. Napisati program
koji izračunava i ispisuje prosečnu vrednost cena u radnji.
\end{primer}

\begin{primer}
U prodavnici se nalaze artikala čije cene su realni pozitivni
brojevi. Program unosi cene artikala sve do unosa broja nula 0. Napisati program
koji izračunava i ispisuje prosečnu vrednost cena u radnji.
\end{primer}
